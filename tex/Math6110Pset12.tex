\documentclass[6pt]{article}
\usepackage[tmargin=1in,bmargin=1in,lmargin=1.5in,rmargin=1.5in]{geometry}
\usepackage{amsfonts,amsmath,amssymb,amsthm,relsize,fancyhdr,parskip,graphicx}

\pagestyle{fancy}
\lhead{Ben Carriel}
\chead{Math 6110 Problem Set 11}
\rhead{\today}

\parskip 7.2pt
\parindent 8pt

\DeclareMathOperator{\N}{\mathbb{N}}
\DeclareMathOperator{\Z}{\mathbb{Z}}
\DeclareMathOperator{\Q}{\mathbb{Q}}
\DeclareMathOperator{\R}{\mathbb{R}}
\DeclareMathOperator{\C}{\mathbb{C}}
\DeclareMathOperator{\capchi}{\raisebox{2pt}{$\mathlarger{\mathlarger{\chi}}$}}

\DeclareMathOperator{\divides}{\mathrel{|}}
\DeclareMathOperator{\suchthat}{\mathrel{|}}

\DeclareMathOperator{\lra}{\longrightarrow}
\DeclareMathOperator{\into}{\hookrightarrow}
\DeclareMathOperator{\onto}{\twoheadrightarrow}
\DeclareMathOperator{\bijection}{\leftrightarrow}

\newcommand{\problem}[1]{\noindent{\textbf{Problem #1}}\\}
\newcommand{\problempart}[1]{\noindent{\textbf{(#1)}}}

\newcommand{\der}[2]{\frac{\partial #1}{\partial #2}}
\newcommand{\norm}[1]{\|#1\|}
\newcommand{\diam}[1]{\text{diam}(#1)}

\DeclareMathOperator{\im}{\text{im}}

\newtheorem*{thm}{\\ Theorem}
\newtheorem*{lem}{\\ Lemma}
\newtheorem*{claim}{\\ Claim}
\newtheorem*{defn}{\\ Definition}
\newtheorem*{prop}{\\ Proposition}

\begin{document}

\problem{4.6.1} \problempart{a} We need to show that $U = \bigcap_n
U_n$ is a generic set of measure 0. To see that $U$ is generic we will
show that $U^c$ is meager. Indeed, we have that $U^c = \bigcup_n
U_n^c$. Each of the $U_n$ is open and dense because it can be written
as the union of intervals about the rationals, which are
dense. Consequently, we have that $U_n^c$ is closed and meager. But
then $U^c$ can be written as the countable union of meager sets, and
is therefore meager.

To see that $U$ has measure $0$ we observe that
\[
m(U) \leq m(U_n) = \frac{1}{n} \sum_{j=1}^\infty 2^{j-1} = \frac{2}{n}
\]
for every $n$. And therefore $m(U) = 0$.

\problempart{b} Recall the construction on the Cantor-like set. At
each iteration of the construction we remove a centrally situated
interval of length $\ell_k$ from each remaining interval in the
construction subject to the restriction that $\sum_{k=1}^\infty \ell_k
2^{-k} < 1$. If we let $\mathcal{C}$ denote a Cantor-like set, we are
tasked to show that $\mathcal{C}$ is first category. Suppose not, then
we have that the $\text{Int}(\overline{\mathcal{C}})$ contains an
interval. We know that $\mathcal{C}$ is closed so $\mathcal{C} =
\overline{\mathcal{C}}$ and therefore we have that
$\text{Int}(\mathcal{C})$ contains an interval. This is impossible by
Book III Exercise 1.6.4 (proved in problem set 1) and so we have that
$\mathcal{C}$ is first category.

\problem{4.6.2}
\problempart{a} In the forward direction we have that $F$ is closed and first category. This means that we can write 
\[
F = \bigcup_{n=1}^\infty E_n
\]
To see that $\text{Int}(F) = \emptyset$ we suppose to the contrary. Then we must be able to find some closed ball $\overline{B} \subset F$. We then note that $\overline{B}$ is a complete metric space in its own right and apply the category theorem to get that $\overline{B}$ is second category in itself. Hence, is non-empty and open. If $x \in \text{Int}(\overline{B}$ then we can find some open ball about $x$ of radius $\delta$ that is wholly contained in $\text{Int}(\overline{B}$. But then we have that 
\[
\overline{B_\delta(x)} = \bigcup_{n=1}^\infty (\overline{B_\delta(x)} \cap E_n)
\]
And so $E_n \cap \overline{B_\delta(x)}$ must be non-empty for some $n$. This implies that the interior of one of the $E_n$ is non-empty and therefore $E_n$cannot be nowhere dense. This contradiction proves the forward direction.

In the reverse direction we have that $F$ is closed and has empty interior. Hence $\text{Int}(\overline{F}) = \text{Int}(F) = \emptyset$ and so $F$ is nowhere dense and therefore first category. 
 
\problempart{b} For this part we appeal to part $\textbf{(a)}$. In the forward direction we must have that $\mathcal{O}$ is open and first category. If we write $\mathcal{O} = \bigcup_n E_n$ with each of the $E_n$ nowhere dense. then we must have that $\mathcal{O}^c = \bigcap_n E_n^c$ is a closed, second category set. Furthermore, each on the $E_n^c$ is dense and so we have that $\mathcal{O}^c$ is dense as well. Then $\mathcal{O}^c$ is a closed, dense set in $X$ and therefore equals the whole of $X$. So $(\mathcal{O}^c)^c = \mathcal{O} = \emptyset$. The reverse direction is trivial because the empty set is its own closure and therefore it is nowhere dense and consequently first category. 

\problempart{c} Suppose that $F$ is generic. Then we have that $F^c$ must be meager and this happens iff $F^c = \emptyset$ because $F^c$ is open, which implies that $F = X$. The reverse direction is the category theorem. 

If $\mathcal{O}$ is generic, then we have that $\mathcal{O}^c$ is meager and therefore has empty interior. The reverse direction is the same, we have that $\mathcal{O}^c$ having no interior means that it is of first category. So its complement, $\mathcal{O}$ is of the second category. 

\problem{4.6.5}
\problempart{a} If $Y$ is a dense $G_\delta$ set then we have that $Y = \bigcap_{n=1}^\infty U_n$ where each of the $U_n$ is open and dense (otherwise $Y$ could not be dense). We need to show that $Y$ is generic. Suppose to the contrary that $Y$ is first category and so we also have that $Y = \bigcup_{n=1}^\infty W_n$ where each of the $W_n$ is nowhere dense. We then note that each of the sets $U_n^c$ is nowhere dense and then write
\[
X = Y^c \cup Y = \bigcup_{n=1}^\infty U_n^c \cup \bigcup_{n=1}^\infty W_n
\] 
But then $X$ is a countable union of nowhere dense sets, and so it is of first category. But this violates the Baire theorem and so we must have had $Y$ was generic. 

\problempart{b} We can write any countable dense set as a union of singletons, each of which are closed and nowhere dense. Hence, any countable dense set is an $F_\sigma$. However, the previous part says that it cannot be a $G_\delta$ because if it were then it would be generic, which is not possible. 

\problempart{c} Let $E$ be a generic set in $X$. Then we have that $E^c$ is of the first category. So we can write $E^c = \bigcup_{n=1}^\infty W_n$ with each $W_n$ nowhere dense. Hence, we can write $E = \bigcup_{n=1}^\infty W_n^c$ and each of the $W_n^c$ is dense. We then set $E_0 = \cap_{n=1}^\infty \text{Int}(W_n^c)$. Clearly, $E_0 \subset E$ because $\text{Int}(W_n^c \subset W_n^c$. Additionally, each of the $\text{Int}(W_n^c)$ is an open dense set. Hence, $E_0$ is a dense $G_\delta$ contained in $E$.

\problem{4.6.6} The hard work for this fact was done in the proof of Theorem 4.1.3. They showed that the set 
\[
E_\epsilon = \{x \suchthat \text{osc}(f)(x) < \epsilon\}
\]
is open. This fact, coupled with the observation (also shown in the text) that $\text{osc}(f)(x) = 0$ iff $f$ is continuous at $x$ gives that we can always write the set of points of continuity of $f$, say $\mathcal{C}_f$ as
\[
\mathcal{C}_f = \bigcap_{n=1}^\infty E_{1/n}
\]
This is a $G_\delta$ set. We then apply the previous exercise to the countable, dense set $\Q$ to see that it cannot be a $G_\delta$ and therefore cannot be the set of continuity points of any function $\R \to \R$. 

\problem{4.6.8} Suppose towards a contradiciton that the Banach space $X$ had a countable Hamel basis say $\{f_n\}_{n=1}^\infty$. If we let $S_m = \text{span}(f_{k_1}, f_{k_2},\ldots , f_{k_m}$ be some finite dimensional subspace, then we can see that $S$ must be closed in $X$. Moreover, we can see that a finite dimensional subspace $S$ has non-empty interior iff $S = X$. Indeed, it $S = X$ then $\text{Int}(S) = X$. Additionally, observe that if $\text{Int}(S) \neq \emptyset$ then there would be some $s \in \text{Int}(S)$. Let $\delta$ be such that $\overline{B_\delta(s)} \subset S$. Then we have that $\overline{B_\delta(0)} = \overline{B_\delta(s)} - s \subset S$. Let $x \in X$ so that if $x = 0$ then $x \in S$ and otherwise $\frac{\delta x}{\norm{x}} \in \overline{B_\delta(0)} \subset S$ and so $x \in S$ as well.  

We have that $S_m^c$ is open in $X$. Since $f_N$ is not in $S_m$ for sufficiently large $N$ we see that $S_m \neq X$ and hence $\text{Int}(S_m)$ is empty by the above. This means that $S_m^c$ must be dense in $X$. Because $X$ is complete, we have that 
\[
D = \bigcap_{m=1}^\infty S_m^c
\]
must be dense in $X$ by the category theorem. However, because the span of the $f_i$ is all of $X$ we have that $D = \emptyset$ and we have a contradiction. So $X$ cannot have a countable basis. 

\problem{4.6.12} For each $x \in X$ we define the linear operator $B_x(y) = T(x,y)$ and similarly we have that $B_y(x) = T(x,y)$. Because each of these functions is continuous, both of these operators are bounded. Hence,
\[
\norm{T(x,y)} \leq C_y\norm{x}
\]
In terms of $B_x$ we have that 
\[
\norm{B_x} \leq C_y\norm{x}
\]
Hence, we have that the family of norms $\{\norm{B_x(y)} \suchthat x \in X, \norm{x} = 1\}$ is bounded for each fixed $y$. In the same way he have that the $\{\norm{B_y(x)} \suchthat y \in Y, \norm{y} = 1\}$ is bounded in the norm. this gives that $\norm{T(x,y)} \leq C_xC_y$ for each $x$ and $y$. Then we apply the uniform boundedness principle to get that $\norm{T(x,y)} \leq C$ for all $x,y$. By scaling the unit vectors we get that $\norm{T(x,y)} \leq C\norm{x}\norm{y}$. 

\problem{4.6.13} We define the sequence of linear operators
\[
\ell_n(g) = \int_X f_n(x)g(x)d\mu (x)
\]
We apply the uniform boundedness principle to see that $\{\ell_n\}_{n=1}^\infty$ is either uniformly bounded, or unbounded on a $G_\delta$ set. If the sequence $f_n$ converges weakly (or is weakly bounded) then for a fixed $g$ we must have that
\[
\lim_{n\to\infty} |\ell_n(g)| = \lim_{n \to \infty} \left|\int_X f_n(x)g(x)d\mu (x)\right|< \infty
\] 
in particular the limit must exist. A convergent sequence of reals is bounded and so the set of $g$ for which 
\[
\sup_n \int_X f_n(x)g(x)d\mu (x) = \infty
\]
is empty. Hence, we must have a uniform bound in the norm. This means that 
\[
\norm{\ell_n} = \sup_{g \neq 0} \frac{|\ell_n(g)|}{\norm{g}} < M, \forall n
\]
By Cauchy Schwartz we immediately have that 
\[
\sup_n \norm{f_n}_{L^p} < \infty 
\]

\end{document}