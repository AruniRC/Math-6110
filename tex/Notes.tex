\documentclass{article}
\usepackage[tmargin=1in,bmargin=1in,lmargin=1.5in,rmargin=1.5in]{geometry}
\usepackage{amsfonts,amsmath,amssymb,amsthm,relsize,fancyhdr,parskip,graphicx}

\pagestyle{fancy}
\lhead{Ben Carriel}
\chead{Math 6110 Notes}
\rhead{\today}

\parskip 7.2pt
\parindent 8pt

\DeclareMathOperator{\N}{\mathbb{N}}
\DeclareMathOperator{\Z}{\mathbb{Z}}
\DeclareMathOperator{\Q}{\mathbb{Q}}
\DeclareMathOperator{\R}{\mathbb{R}}
\DeclareMathOperator{\C}{\mathbb{C}}
\DeclareMathOperator{\capchi}{\raisebox{2pt}{$\mathlarger{\mathlarger{\chi}}$}}

\DeclareMathOperator{\divides}{\mathrel{|}}
\DeclareMathOperator{\suchthat}{\mathrel{:}}

\DeclareMathOperator{\lra}{\longrightarrow}
\DeclareMathOperator{\into}{\hookrightarrow}
\DeclareMathOperator{\onto}{\twoheadrightarrow}
\DeclareMathOperator{\bijection}{\leftrightarrow}

\newcommand{\problem}[1]{\noindent{\textbf{Problem #1}}\\}
\newcommand{\problempart}[1]{\noindent{\textbf{(#1)}}}

\newcommand{\der}[2]{\frac{\partial #1}{\partial #2}}
\newcommand{\norm}[1]{\|#1\|}
\newcommand{\diam}[1]{\text{diam}(#1)}

\DeclareMathOperator{\im}{\text{im}}

\newtheorem*{thm}{\\ Theorem}
\newtheorem*{lem}{\\ Lemma}
\newtheorem*{claim}{\\ Claim}
\newtheorem*{defn}{\\ Definition}
\newtheorem*{prop}{\\ Proposition}

\begin{document}
\section{Measure and Integration Theory}
Basically just a collection of Theorems and proofs from measure theory.

\subsection{Measurable Sets}
\begin{thm}[Inner and Outer Regularity for Lebesgue Measure]
  Suppose that $E_1, E_2,\ldots$ is a sequence of measurable sets in $\R^d$
  then
  \begin{enumerate}
  \item[I.] If $E_k \nearrow E$ then $\lim_{k\to \infty} m(E_k) = m(E)$.
  \item[I.] If $E_k \searrow E$ and $m(E_k) < \infty$ for some $k$ then
    $\lim_{k\to \infty} m(E_k) = m(E)$.
  \end{enumerate}
\end{thm}
\begin{proof}
  For $I.$ we let $F_1 = E_1$ and then recursively define $F_n = E_n - E_{n-1}
  $. By construction we have that each of the $F_n$ are measurable, disjoint
  and $\bigcup_{n}F_n = E$. So we have that
  \[
  m(E) = \sum_{n=1}^\infty m(F_n) = \lim_{N\to\infty}\sum_{n=1}^N m(F_n) =
  \lim_{N\to\infty} m\left(\bigcup_{n=1}^\infty F_n\right) = \lim_{N\to\infty}
  m(E_n)
  \]
  So we have the desired limit.

  For $II.$ we suppose without loss of generality that $m(E_1) < \infty$. We
  then do a similar construction letting $F_n = E_n - E_{n+1}$ so that we
  have
  \[
  E_1 = E \cup \bigcup_{n=1}^\infty F_n
  \]
  is a disjoint union of measurable sets. We then see that
  \begin{align*}
    m(E_1) &= m(E) + \lim_{N\to\infty}\sum_{n=1}^{N-1}m(E_n - E_{n+1} \\
    &= m(E) + m(E_1) - \lim_{N\to\infty} m(E_N)
  \end{align*}
  Subtracting $m(E_1)$ from both sides gives the result. 
\end{proof}

\begin{thm}[Regularity Theorem for Lebesgue Measure]
  Suppose that $E \subset \R^d$ is measurable. Then for every $\epsilon > 0$
  \begin{enumerate}
  \item[I.] There exists an open set $\mathcal{O}$ with $E
    \subset \mathcal{O}$ and $m(\mathcal{O} - E) \leq \epsilon$.
  \item[II.] There exists a closed set $F$ with $F \subset E$ and $m(E-F)
    \leq \epsilon$.
  \item[III.] If $m(E) < \infty$, there exists a compact set $K$ with $K
    \subset E$ and $m(E-K) \leq \epsilon$.
  \item[IV.] If $m(E) < \infty$, there exists a finite union $F =
    \bigcup_{j=1}^N Q_j$ of closed cubes such that $m(E\Delta F) \leq
    \epsilon$. Where $\Delta$ is the summetric difference operator.
  \end{enumerate}
\end{thm}
\begin{proof}
  The proof of $I.$ is just the definition of measurability we chose (not
  Carath\'{e}odory measurability). To show $II.$ we reduce to $I.$
  because if $E$ is measurable then $E^c$ is measurable and so there is some
  open set $\mathcal{O}$ that contains $E^c$ and has $m(\mathcal{O}-E^c)\leq
  \epsilon$. We then note that $F = \mathcal{O}^c$ is closed and contained in
  $E$. Moreover, $E-F = \mathcal{O} - E^c \leq \epsilon$. So $II.$ is proved.

  For $III.$ we use the usual trick to go from closed sets to compact sets on
  $\sigma$-finite measure spaces. We know by $II.$ that there is a closed set
  $F \subset E$ such that $m(E-F) \leq \epsilon$. Define the decreasing
  sequence of sets $K_n = E \cap B_n$ where $B_n$ is the ball of radius $n$
  centered at the origin. Then we have that $E - K_n$ is a decreasing
  sequence of measurable sets that decreases to $E - F$ and so we have that
  for sufficiently large $n$ that $m(E - K_n) \leq \epsilon$. Note that it is
  critical that $m(E)$ is finite.

  For the proof of $IV.$ we choose a family of closed cubes $Q_j$ such that
  \[
  E \subset \bigcup_{j=1}^\infty Q_j \text{ and } \sum_{j=1}^\infty |Q_j| \leq
  m(E) + \epsilon/2
  \]
  Because $m(E) < \infty$ we have that the series converges and so we can
  find $N > 0$ such that $\sum_{j=N+1}^\infty |Q_j| < \epsilon/2$. If we set
  $F = \bigcup_{j=1}^N Q_j$ then we see that
  \begin{align*}
    m(E\Delta F) &= m(E-F) + m(F-E) \\
    &\leq m\left( \bigcup_{N+1}^\infty Q_j\right) + m\left(\bigcup_{j=1}^\infty
      Q_j - E\right) \\
    &\leq \sum_{j=N+1}^\infty |Q_j| + \sum_{j=1}^\infty |Q_j| - m(E) \\
    &\leq \epsilon
  \end{align*}
  So we are done.
\end{proof}

\begin{thm}[Decomposition of Measurable Sets]
  A subset $E$ in $\R^d$ is measurable
  \begin{enumerate}
  \item[I.] if and only if $E$ differs from a $G_\delta$ set by a set of
    measure zero.
  \item[II.] if and only if $E$ differs from an $F_\sigma$ set by a set of
    measure zero.
  \end{enumerate}
\end{thm}
\begin{proof}
  The forward direction is trivial because we know that if $E = A \cup B$
  where $A$ is either $F_\sigma$ or $G_\delta$ and $B$ has measure zero, then
  $E$ is the union of measurable sets and hence, measurable.

  In the reverse direction in $I.$ we are given that $E$ is measurable. For
  each $n$ we know that there is an open set $\mathcal{O}_n$ that contains
  $E$ and such that $m(\mathcal{O}_n - E) \leq 1/n$. Then we set $G =
  \bigcap_{n=1}^\infty \mathcal{O}_n$ is a $G_\delta$ set and we have that
  $(S - E) \subset (\mathcal{O}_n - E)$ for all $n$ so $S - E$ has measure
  zero.

  The reverse direction for $II.$ is very similar. We use the Regularity
  Theorem to get a sequence of closed sets $\mathcal{C}_n$ such that each is
  contained in $E$ and $m(E - \mathcal{C}_n) \leq 1/n$. We set $F =
  \bigcup_{n=1}^\infty \mathcal{C}_n$. Then $F$ is clearly an $F_\sigma$ and
  moreover $m(E-\mathcal{C}_n) \to 0$ as $n\to \infty$ and so $m(E - F) = 0$.
\end{proof}

\subsection{Measurable Functions}

\begin{thm}[Properties of Measurable Functions]
  We have the following:
  \begin{enumerate}
  \item[P1.] The finite- valued function $f$ is measurable iff for
    every open set $\mathcal{O}$ and every closed set $F$ the sets $f^{-1}
    \mathcal{O}$ and $f^{-1}(F)$ are measurable.
  \item[P2.] If $f$ is continuous on $\R^d$, then $f$ is measurable.
    If $f$ is measurable and finite valued and $\Phi$ is continuous then
    $\Phi \circ f$ is measurable.
  \item[P3.] If $\{f_n\}_{n=1}^\infty$ is a sequence of measurable
    functions then
    \[
    \sup_n f_n(x), \inf_n f_n(x), \limsup_{n\to\infty}f_n(x), \text{ and }
    \liminf_n f_n(x)
    \]
    are measurable.
  \item[P4.] If $\{f_n\}_{n=1}^\infty$ is a sequence of measurable
    functions and $f_n \to f$ pointwise a.e. then $f$ is measurable.
  \item[P5.] If $f,g$ are measurable and finite valued then $f+g, fg,
    \text{ and } f^k$ are measurable.
  \item[P6.] If $f$ is measurable and $f = g$ a.e. then $g$ is
    measurable.
  \end{enumerate}
\end{thm}
\begin{proof}
  The proof for $P1$ is clear. If $f$ is measurable then we know that the
  sets $f^{-1}((-\infty, a))$ are measurable for any $a$. We can then
  construct every interval $(a,b)$ by taking $(-\infty, b) \cap
  \bigcup_{n=1}^\infty (a-1/n, \infty)$. This is a countable union of open
  sets and hence open. Furthermore, its inverse image is measurable. We then
  use the fact that the intervals are a basis for the topology on $\R$ to
  complete the proof.

  For $P2$ we use the fact that if $f$ is continuous then the inverse image
  of an open set is open, hence measurable. To see that $\Phi$ is measurable
  we note that it is continuous. Then we have that if $f$ is measurable that
  $(\Phi \circ f)^{-1} = f^{-1} \circ \Phi^{-1}$ is clearly measurable because
  the inverse image of the interval under $\Phi$ is open and $f$ is
  measurable.

  To prove $P3$ we first note that $\{\sup_n f_n(x) > a\} = \bigcup_n
  \{f_n(x) > a\}$ is clearly measurable as the union of measurable sets.
  Similarly, $\inf_n f_n(x) = -\sup_n -f_n(x)$ is measurable. To get
  $\limsup$ and $\liminf$ we simply note that
  \[
  \limsup_{n\to \infty} f_n(x) = \inf_k\{\sup_{n \geq k} f_n(x)\} \text{ and }
  \liminf_{n\to \infty} f_n(x) = \sup_k\{\inf_{n \geq k} f_n(x)\}
  \]

  $P4$ is simply a consequence of $P3$ because we have that
  \[
  f(x) = \limsup_{n\to\infty} f_n(x) = \liminf_{n\to\infty} f_n(x)
  \]

  For $P 5$ we can get that $f+g$ is measurable because we can write
  \[
  \{(f+g)(x) > a\} = \bigcup_{r \in \Q} \{f(x) > a - r\} \cap \{g(x) > r\}
  \]
  which is the union of measurable sets. We can then see that $f^k$ is
  measurable because we have
  \[
  \{f^(x) > a\} =
  \begin{cases}
    \{f(x) > a^{1/k}\} & k \text{ is even} \\
    \{f(x) > a^{1/k}\} \cup \{f(x) < -a^{1/k}\} & k \text{ is odd}
  \end{cases}
  \]

  Finally, for $P6$ we simply note that $f(x) = g(x)$ a.e. implies that the
  sets $\{f(x) < a\}$ and $\{g(x) < a\}$ differ by at most a set of measure
  zero. This immediately gives the result.
\end{proof}

\begin{thm}[Egorov]
  Suppose $\{f_k\}_{k=1}^\infty$ is a sequence of measurable functions defined
  on a measurable set $E$ with $m(E) < \infty$, and assume that $f_k \to f$
  a.e. on $E$. Given $\epsilon > 0$ there exists a closed set $A_\epsilon
  \subset E$ such that $m(E - A_\epsilon) < \epsilon$ and $f_k \to f$
  uniformly on $A_\epsilon$.
\end{thm}
\begin{proof}
  We assume without loss of generality that $f_k \to f$ on all of $E$ because
  we can just replace $E$ by the set $E - N$ if $N$ are the points where
  $f_n$ doesn't converge.

  For each pair of integers of non-negative integers define
  \[
  E_k^n = \{x \in E \suchthat |f_j(x) - f(x)| < 1/n, \text{ for all } j > k\}
  \]
  If we fix $n$ and note that $E_k^n \subset E_{k+1}^n$ and $E_n^n \nearrow
  E$ as $k \to \infty$. We then note that there is some $k_n$ such that
  $m(E - E_{k_n}) < 2^{-n}$. Then we have that
  \[
  |f_j(x) - f(x)| < 1/n \text{ whenever } j > k_n \text{ and } x \in E_{k_n}^n
  \]
  Now we choose $N$ large enough that $\sum_{n=N}^\infty 2^{-n} < \epsilon/2$,
  and let
  \[
   \tilde{A_\epsilon} = \bigcap_{n\geq N} E_{k_n}^n
  \]
  Observe that
  \[
  m(E - \tilde{A_\epsilon}) \leq \sum_{n=N}^\infty m(E - E_{k_n}^n) < \epsilon/2
  \]
  then if we have a $\delta > 0$ then we choose $n \geq N$ such that $1/n <
  \delta$ and note that $x\in \tilde{A_\epsilon}$ implies that $x \in E_{k_n}^n
  $. Thus, $|f_j(x) - f(x)| < \delta$ whenever $j > k_n$. So $f_k \to f$
  uniformly on $\tilde{A_\epsilon}$. We then find a closed subset $A_\epsilon
  \subset \tilde{A_\epsilon}$ such that $m(\tilde{A_\epsilon} - A_\epsilon) <
  \epsilon/2$. Consequently, we see that $m(E - A_\epsilon) < \epsilon$ and so
  we are done.
\end{proof}

\begin{thm}[Lusin]
  Suppose that $f$ is measurable and finite-valued on $E$ with $m(E) < \infty
  $. Then for any $\epsilon > 0$ there exists a closed set $F_\epsilon$ such
  that $F_\epsilon \subset E$ and $m(E - F_\epsilon) \leq \epsilon$ and such
  that $f|_{F_\epsilon}$ is continuous.
\end{thm}
\begin{proof}
  Find a sequence of step function $\{f_n\}$ such that $f_n \to f$ a.e.
  Then we may find sets $E_n$ such that $m(E_n) < 2^{-n}$ and $f_n$ is
  continuous outside of $E_n$. We then apply Egorov's theorem to get a
  set $A_{\epsilon/3}$ such that $f_n \to f$ uniformly on $A_{\epsilon/3}$ and
  $m(E - A_{\epsilon/3}) \leq \epsilon/3$. Then we consider the set
  \[
  \tilde{F} = A_{\epsilon/3} - \bigcup_{n\geq N}E_n
  \]
  for $N$ large enough that $\sum_{n\geq N} 2^{-n} < \epsilon/3$. Now we have
  that for every $n \geq N$ the function $f_n$ is continuous on $\tilde{F}$.
  Thus, $f$ is continuous on $\tilde{F}$ because $f_n \to f$ uniformly there.
  To complete the proof we simply find a closed set $F_\epsilon \subset
  \tilde{F}$ such that $m(\tilde{F} - F_\epsilon) < \epsilon/3$.
\end{proof}

\subsection{Abstract Measure Theory}
\begin{thm}[Carath\'{e}odory]
  Given an outer measure $\mu_*$ on a set $X$, the collextion $\mathfrak{M}$
  of Carath\'{e}odory measurable sets forms a $\sigma$-algebra. Moreover,
  $\mu_*$ restriced to $\mathfrak{M}$ is a measure.
\end{thm}
\begin{proof}
  Clearly, $\emptyset$ and $X$ are in $\mathfrak{M}$ becuase the definition
  of measurability is symmetric in complements.

  Next we need to prove that we are closed under finite unions of disjoint
  sets and that $\mu_*$ is finitely additive on $\mathfrak{M}$. If we have
  $E_1, E_2 \in \mathfrak{M}$ and $A$ is any subset of $X$ then
  \begin{align*}
    \mu_*(A) &= \mu_2(E_2 \cap A) + \mu_*(E_2^c \cap A) \\
    &= \mu_*(E_1 \cap E_2 \cap A) + \mu_*(E_1^c \cap E_2 \cap A) +
    \mu_*(E_1 \cap E_2^c \cap A) + \mu_*(E_1^c \cap E_2^c \cap A) \\
    &\geq \mu_*((E_1 \cup E_2) \cap A) + \mu_*((E_1 \cup E_2)^c \cap A)
  \end{align*}
  The first line follows from the measurability condition on $E_1$ and $E_2$
  and the last inequality is the sub-additivity of $\mu_*$. Therefore we
  have that $E_1 \cup E_2 \in \mathfrak{M}$. If they are disjoint then we
  have that
  \[
  \mu_*(E_1 \cup E_2) = \mu_*(E_1 \cap (E_1 \cup E_2)) + \mu_*(E_1^c \cap
  (E_1 \cup E_2)) = \mu_*(E_1) + \mu_*(E_2)
  \]
  Now we need to show that $\mathfrak{M}$ is closed under countable unions
  of disjoint sets, i.e. $\mu_*$ is countably additive on $\mathfrak{M}$.
  Let $E_1, E_2, \ldots$ be a countable collection of disjoint sets in
  $\mathfrak{M}$. Define
  \[
  G_n = \bigcup_{j=1}^n E_j \text{ and } G = \bigcup_{j=1}^\infty E_j
  \]
  For each $n$ we have that $G_n$ is a finite union of sets in
  $\mathfrak{M}$ and so $G_n \in \mathfrak{M}$. Furthermore, for any
  $A \subset X$ we have that
  \begin{align*}
    \mu_*(G_n\cap A) &= \mu_*(E_n \cap (G_n \cap A)) +
    \mu_*(E_n^c \cap (G_n \cap A)) \\
    &= \mu_*(E_n \cap A) + \mu_*(G_{n-1} \cap A) \\
    &= \sum_{j=1}^\infty \mu_*(E_j \cap A)
  \end{align*}
  The last inequality is gotten by induction. We know that $G_n \in
  \mathfrak{M}$ and $G^c \subset G_n^c$ we see that
  \[
  \mu_*(A) = \mu_*(G_n \cap A) + \mu_*(G_n^c \cap A) \geq \sum_{j=1}^\infty
  \mu_*(E_j \cap A) + \mu_*(G^c \cap A)
  \]
  We then let $n \to \infty$ to see that
  \[
  \mu_*(A) \geq  \sum_{j=1}^\infty \mu_*(E_j \cap A) + \mu_*(G^c \cap A) \geq
  \mu_*(G \cap A) + \mu_*(G^c \cap A) \geq \mu_*(A)
  \]
  Therefore all of the inequalities above must in fact be equalities and so
  we have that $G \in \mathfrak{M}$. Furthermore, if we take $A = G$ in the
  above we get that $\mu_*$ is countably additive on $\mathfrak{M}$ and so
  we are done.
\end{proof}

\begin{thm}[Carath\'{e}odory Extension Theorem]
  Suppose that $\mathcal{A}$ is an algebra of sets in $X$, $\mu_0$ is a
  premeasure on $\mathcal{A}$, $\mathfrak{M}$ is the $\sigma$-algebra
  generated by $\mathcal{A}$. Then there exists a measure $\mu$ on $
  \mathfrak{M}$ that extends $\mu_0$. If $\mu$ is $\sigma$-finite then this
  extension is unique.
\end{thm}
\begin{proof}
  The exterior $\mu_*$ induced by $\mu_0$ defines a measure on the $\sigma$-
  algebra of Carath\'{e}odory measurable sets defined by
  \[
  \mu_*(E) = \inf \left\{ \sum_{j=1}^\infty \mu_0(E_j) \suchthat E \subset
    \bigcup_{j=1}^\infty E_j, \text{ where } E_j \in \mathcal{A} \text{ for
       all } j\right\}
  \]
  (The proof is clear. The $E_j$ play the role of cubes). Hence, $\mu$ is
  also a measure on $\mathfrak{M}$ that extends $\mu_0$.

  To show uniqueness whenever $\mu$ is $\sigma$-finite on the $\sigma$-
  algebra of Carath\'{e}odory measurable sets we suppose that there is
  another measure $\nu$ that agrees with $\mu_0$ on $\mathcal{A}$. Suppose
  that $F \in \mathfrak{M}$ has finite measure. Then we see
  \[
  \nu(F) \leq \sum_{j=1}^\infty \nu(E_j) = \sum_{j=1}^\infty \mu_0(E_j)
  \]
  So we get for free that $\nu(F) \leq \mu(F)$. Now we need the reverse
  inequality. Observe that if $E = \bigcup_j F_j$ then then fact that $\nu$
  and $\mu$ agree on $\mathcal{A}$ gives that
  \[
  \nu(E) = \lim_{n\to\infty}\nu\left(\bigcup_{j=1}^n E_j\right) = \lim_{n\to
    \infty} \mu\left(\bigcup_{j=1}^n E_j\right) = \mu(E)
  \]
  If we pick the $E_j$ such that $\mu(e) \leq \mu(F) + \epsilon$, then the
  fact that $\mu(F) < \infty$ implies that $\mu(E-F) \leq \epsilon$. So
  \[
  \mu(F) \leq \mu(E) = \nu(E) = \nu(F) + \nu (e-F) \leq \nu(F) + \mu(E-F)
  \leq \mu(F) + \epsilon
  \]
  Since $\epsilon$ is arbitrary we have the reverse inequality is proved.

  Now we can use the $\sigma$-finiteness. We write $X = \bigcup_j E_j$ where
  $E_1, E_2,\ldots$ is a countable collection of disjoint sets in $
  \mathcal{A}$ with finite measure. Then for any $F \in \mathfrak{M}$ we have
  that
  \[
  \mu(F) = \sum_{j=1}^\infty \mu(F\cap E_j) = \sum_{j=1}^\infty \nu(F\cap E_j) =
  \nu(F)
  \]
\end{proof}

\subsection{Integration}
\begin{thm}[Monotone Convergence Theorem]
  Let $\{f_n\}$ be a sequence of measurable functions on $X$ and suppose that
  \begin{enumerate}
  \item $f_n(x) \geq 0$ for every $x$ in $X$.
  \item $f_n(x) \to f(x)$ for every $x \in X$.
  \end{enumerate}
  Then $f$ is measurable and
  \[
  \lim_{n\to \infty} \int_X f_n d\mu = \int_X f d\mu
  \]
\end{thm}
\begin{proof}
  Since we know that $\int f_n d\mu \leq \int f_{n+1} d\mu$ we have that
  there exists an $\alpha \in [0, \infty]$ such that
  \[
  \lim_{n\to\infty}\int f_n d\mu \to \alpha
  \]
  Since $f_n \leq f$ for every $n$ we see $\int f_nd\mu \leq \int fd\mu$ for
  every $n$. So the above implies that $\alpha \leq \int f d\mu$. Let $s$ be
  sny simple measurable sunction such that $0 \leq s \leq f$ and let $c$ be
  a constant such that $ 0 < c < 1$ and define
  \[
  E_n = \{x \suchthat f_n(x) \geq cs(x)\}
  \]
  Each $E_n$ is measurable, $E_1 \subset E_2 \subset \cdots$, and $X =
  \bigcup_n E_n$. To see the equality consider a particular $x \in X$. If
  $f(x) = 0$ then $x \in E_1$ and if $f(x) > 0$ then $cs(x) < f(x)$ because
  $c < 1$. Hence, we have that $x \in E_n$ for some $n$. Also,
  \[
  \int_X f_n d\mu \geq \int_{E_n} f_n d\mu \geq c\int_{E_n} s d\mu
  \]
  for every $n$. We then let $n \to \infty$ and so $\alpha \geq c \int_X
  sd\mu$. Since this holds for every $c < 1$ we have that
  \[
  \alpha \geq \int s d\mu
  \]
  For every simple function $s \leq f$. Hence, we have that
  \[
  \alpha \geq \int_X fd\mu
  \]
  This proves the reverse inequality and we are done.
\end{proof}

\begin{thm}[Fatou's Lemma]
  If $\{f_n\}$ is a sequence of non-negative measurable functions on $X$ then
  \[
  \int \liminf_{n\to\infty} f_n d\mu \leq \liminf_{n\to\infty} \int f_n d\mu
  \]
\end{thm}
\begin{proof}
  Let $g_k(x) = \inf_{i \geq k} f_i(x)$. Then we have that $g_k \leq f_k$, so
  we see that
  \[
  \int g_kd\mu \leq \int f_k d\mu
  \]
  Moreover we see that $0 \leq g_1 \leq g_2 \leq \cdots $, and each $g_k$ is
  measurable with $g_k(x) \to \lim\inf f_n(x)$ as $k \to \infty$. The
  Monotone Convergence Theorem implies that the left side tends to the right
  side as $k \to \infty$ we are done.
\end{proof}

\begin{thm}[Dominated Convergence Theorem]
  Suppose that $\{f_n\}$ is a sequence of complex measurable functions on $X$
  such that
  \[
   \lim_{n\to\infty} f_n(x) = f(x)
  \]
  exists for every $x \in X$. If there is a function $g \in L^1(\mu)$ such
  that $|f_n(x)| \leq g(x)$ for each $x \in X$ and all $n > 0$ then $f \in
  L^1(\mu)$ and
  \[
  \lim_{n\to\infty} \int_X |f_n - f| d\mu = 0
  \]
  as well as
  \[
  \lim_{n\to\infty} \int_X f_n d\mu = \int_X (\lim_{n\to \infty} f_n) d\mu
  \]
\end{thm}
\begin{proof}
  It is clear that $f \in L^1(\mu)$ because $|f| \leq g$ and $f$ is
  measurable. We also have that $|f_n - f| \leq 2g$ and so we can apply
  Fatou's lemma to then function $2g - |f_n - f|$ to see that
  \begin{align*}
    \int_X 2g d\mu &\leq \liminf_{n\to\infty}\int_X (2g - |f_n - f|) d\mu \\
    &= \int_X 2g d\mu + \liminf_{n\to\infty}\left(-\int_X|f_n - f| d\mu\right)
    \\ &= \int_X 2g d\mu - \limsup_{n\to\infty}\int_X |f_n - f|d\mu
  \end{align*}
  Becuase $\int_X 2gd\mu$ is finite we can subtract it to see that
  \[
  \limsup_{n\to\infty} \int_X |f_n - f|d\mu \leq 0
  \]
  This immediately gives that
  \[
  \lim_{n\to\infty}\int_X |f_n -f| d\mu = 0
  \]
  We then recall that if $f \in L^1$ then $|\int_x f| \leq \int_X|f|$ so we
  get
  \[
  \lim_{n\to\infty} \int_X f_n d\mu = \int_X fd\mu
  \]
\end{proof}

\begin{thm}[Fubini's Theorem]
  Suppose that $\mu_1,\mu_2$ are $\sigma$-finite complete measures on
  $X_1$ and $X_2$, respectively and $f(x_1,x_2)$ is integrable on
  $(X_1 \times X_2, \mu_1 \times \mu_2)$ and is such that for every
  measurable set $E$ we have that the slice $E^{x_2}$ is
  $\mu_1$-measurable and $\mu_1(E^{x_2})$ is defined for almost every
  $x_2 \in X_2$. Then
  \begin{enumerate}
  \item For almost every $x_2 \in X_2$, the slice $f^{x_2}(x_1) =
    f(x_1,x_2)$ is integrable on $(X_1, x_1)$.
  \item $\int_{X_1}f(x_1,x_2)d\mu_1$ is an integrable function on
    $X_2$.
  \item \[\int_{X_2}\left(\int_{X_1}f(x_1,x_2)d\mu_1\right)d\mu_2 =
    \int_{X_1\times X_2}f(x_1,x_2)d\mu_1\times \mu_2\]
  \end{enumerate}
\end{thm}
\begin{proof}
  This one is a marathon. We have to start by building up the product
  measure . We begin with measurable rectangles $A \times B $ in $X_1
  \times X_2$ as cartesian products of measurable sets and define the
  function
  \[
  \mu_0(A\times B) = \mu_1(A)\mu_1(B)
  \]
  It is clear that this is a premeasure on the algebra $\mathcal{A}$
  of sets generated by the measurable rectangles (use the Monotone
  Convergence Theorem). We now show the following
  \begin{lem}
    If $E$ belongs to $\mathcal{A}_{\sigma\delta}$, then $E^{x_2}$ is
    $\mu_1$- measurable for every $x_2$. Moreover, $\mu_1(E^{x_2})$ is
    a $\mu_2$- measurable function and
    \[
    \int_{X_2}\mu_1(E^{x_2})d\mu_2 = (\mu_1\times \mu_2)(E)
    \]
  \end{lem}
  \begin{proof}
    First note that this is trivial when $E$ is a measurable
    rectangle. Next, suppose that $E$ is a set in
    $\mathcal{A_\sigma}$. Then we can decompose it into a countable
    collection of disjoint rectangles $E_j$. Then for each $x_2$ we
    have that $E^{x_2} = \bigcup_j E_j^{x_2}$, and then we observe
    that each of the $E_j^{x_2}$ are disjoint as well. Thus, we have
    that
    \[
    \int_{X_2}\mu_1(E_j^{x_2})d\mu_2 = (\mu_1\times \mu_2)(E_j)
    \]
    We then apply the Monotone Convergence Theorem to get the result
    for each $E \in \mathcal{A}_\sigma$.

    Next, we take $E \in \mathcal{A}_{\sigma\delta}$ such that
    $(\mu_1\times \mu_2)(E)<\infty$. Then there exists a sequence of
    sets $E_j$ in $\mathcal{A_\sigma}$ such that $E_k \subset E_{k+1}$
    $E_k\in\mathcal{A_ \sigma}$ and $E = \bigcap_k E_k$. We then set
    $f_j(x_2) = \mu_1(E_j^{x_2}$ and $f(x_2) = \mu_1(E^{x_2})$. To see
    that $E^{x_2}$ is measurable and $f$ is well-defined observe that
    $E^{x_2}$ is the decreasing limit of the sets $E_j^{x_2}$, which
    are all measurable by the preceding argument.  Also, because $E_1
    \in \mathcal{A_\sigma}$ and $(\mu_1\times\mu_2)(E) < \infty$ we
    see that $f_j(x_2) \to f(x_2)$ as $j\to\infty$ for each $x_2$.
    Thus, we see that $f(x_2)$ is measurable. However, we also see
    that $\{f_j(x_2)\}$ is a decreasing sequence of non-negative
    functions and hence we have that
    \[
    \int_{X_2}f(x_2)d\mu_2(x) = \lim_{j\to\infty}\int_{X_2}f_j(x_2)d\mu_2(x)
    \]
    And therefore we are done in the case that $(\mu_1\times\mu_2)(E)
    < \infty$.

    When $(\mu_1\times\mu_2)(E) = \infty$ we need to use the
    $\sigma$-finiteness of $\mu_1$ and $\mu_2$ to get the result to
    hold.  Because both are $\sigma$-finite we can find sequence
    $\{F_j\}_1^\infty$ and $\{G_k\}_1^\infty$ such that $F_j \subset
    F_{j+1}$ and $G_k \subset G_{k+1}$ such that $\mu_1(F_j),
    \mu_2(G_k) < \infty$ for all $j,k$ and $X_1 = \bigcup_j F_j$ and
    $X_2 = \bigcup_k G_k$. Then we simply replace $E$ by $E \cap (F_j
    \times G_j)$ and let $j\to\infty$ to get the lemma.
  \end{proof}

  Now we extend the previous lemma to arbitrary measurable sets in
  $X_1\times X_2$. If we let $\mathfrak{M}$ be the $\sigma$-algebra
  generated by the measurable rectangles then we see the following

  \begin{lem}
    If $E$ is an arbitrary measurable set in $X$ then the conclusion
    of the previous lemma is still valid with the relaxed hypothesis
    that $E^{x_2}$ is $\mu_1$-measurable and $\mu_1(E^{x_2})$ is
    defined for almost every $x_2 \in X_2$.
  \end{lem}
  \begin{proof}
    Consider first the case of $m(E) = 0$. Then by the previous lemma
    we have that there is some $F \in \mathcal{A}_{\sigma\delta}$ such
    that $E \subset F$ and $(\mu_1 \times \mu_2)(F) = 0$. Since
    $E^{x_2} \subset F^{x_2}$ for every $x_2$ and $F^{x_2}$ has
    $\mu_1$-measure zero for almost every $x_2$ (apply integral in
    previous lemma), the assumed completeness of the measure $\mu_2$
    immediately gives that $E^{x_2}$ is measurable and has measure
    zero for those $x_2$. Thus, we have the right conclusion when $E$
    has measure zero.

    For general $E$ we find an $F \in A_{\sigma\delta}$ such that $F
    \supset E$ and $F- E = Z$ has measure zero (Give a proof that this
    is possible. Not hard). Since $F^{x_2} - E^{x_2} = Z^{x_2}$ we
    apply the case we just proved and observe that for almost all
    $x_2$ the set $E^{x_2}$ is measurable and $\mu_1(E^{x_2}) =
    \mu_1(F^{x_2}) - \mu_1(Z^{x_2})$. This proves the lemma.
  \end{proof}

  Now we have the machinery to get the theorem fairly easily. We note
  that if the desired conclusions hold for finitely many functions,
  then they must also hold for their linear combinations. So we assume
  that $f$ is non-negative (otherwise decompose into positive and
  negative parts). When $f = \capchi_E$, with $E$ a set of finite
  measure, what we need to proved is contained in the previous
  lemma. So the desired result holds for simple functions. Therefore,
  by the Monotone Convergence theorem we have that it holds for all
  non-negative functions.
\end{proof}

\begin{thm}[Polar Coordinates Formula]
  Suppose that $f$ is an integrable function on $\R^d$. then for
  almost every $\gamma$ in $S^{d-1}$ the slice defined by $f^\gamma(r)
  = f(r\gamma)$ is an integrable function with respect to the measure
  $r^{d-1}dr$. Moreover, $\int_0^\infty f^\gamma(r)r^{d-1}dr$ is
  integrable on $S^{d-1}$ and the following holds
  \[
  \int_{\R^d}f(x) = \int_{S^{d-1}}\left(\int_0^\infty
    f(r\gamma)r^{d-1}dr\right)d\sigma(\gamma)
  \]
\end{thm}
\begin{proof}
  We consider the product measure $\mu = \mu_1 \times \mu_2$ on $X_1
  \times X_2$ given by Fubini's theorem \textit{3.} Since the space
  \[
  X_1 \times X_2 = \{(r,\gamma) \suchthat 0<r<\infty \text{ and
  }\gamma\in S^{d-1}\}
  \]
  can be identified with $\R^d - \{0\}$, we can think of $\mu$ as a
  measure on the latter space and our main goal is to identify it with
  the (restriction of) the Lebesgue measure on that space. We first
  have the following
  \begin{claim}
    If $E = E_1 \times E_2$ is a measurable set then we have
    that $m(E) = \mu(E)$ and $\mu(E) = \mu_1(E_1)\mu_1(E_2)$.
  \end{claim}
  \begin{proof}
    This is clear when $E$ is a measurable rectangle $E = E_1 \times
    E_2$ and in this case we see that $\mu(E) =
    \mu_1(E_1)\mu_1(E_2)$. We then extend it to the case that $E_2$ is
    an arbitrary measurable subset of $S^{d-1}$ and $E_1 = (0,1)$ by
    noting that $E$ is the sector of $\tilde{E_2}$, while $\mu_1(E_1) = 1/d$.

    Because of the relative dilation invariance of the Lebesgue
    measure we have the claim for sets $E = (0,b) \times E_2$ and $b >
    0$. We then use a limiting argument to get half-open intervals of
    the form $E_1 = (0, a]$. We can then easily get the open intervals
    $E_1 = (a,b)$ by set subtraction and thus we get all open sets. So
    at this point we have $m(E_1 \times E_2)$ for all open sets $E_1$,
    and hence for all closed sets as well and consequently for all
    Lebesgue measurable sets. So we have established the claim for all
    measurable rectangles and as a result all finite unions of
    measurable rectangles.
  \end{proof}
  This is an algebra of sets and so we apply the Carath\'{e}odory
  Extension theorem to extend this to the $\sigma$-algebra genrated by
  the measurable rectangles. Hence, we have that whenever $E \in
  \mathfrak{M}$, the result of the theorem holds for $\capchi_E$.

  To make the final extension, we note that any open set in $\R^d -
  \{0\}$ can be written as a countable union of rectangles $ \bigcup_j
  A_j \times B_j$, where $A_j$ and $B_j$ are open in $(0,\infty)$ and
  $S^{d-1}$ respectively (The proof of this fact is
  clear. Exercise). It follows that any open set is in $\mathfrak{M}$,
  and therefore so is any Borel set. So we have that the theorem is
  true for $\capchi_E$ whenever $E$ is a Borel set in $\R^d -
  \{0\}$. Then we clearly get the result for all characteristic
  functions of Lebesgue measurable sets because they can be decomposed
  into a Borel set and a set of measure zero. We then complete the
  proof by the familiar construction of going from characteristic
  functions, to simple functions, and then using the Monotone
  convergence theorem to get the general result.
\end{proof}

\begin{thm}[Lebesgue-Stieltjes Integral]
  Let $F$ be an increasing function on $\R$ that is normalized. Then
  there exists a unique measure $\mu$ (also known as $dF$) on the
  Borel sets $\mathcal{B}$ on $\R$ such that $\mu((a,b]) = F(b) -
  F(a)$ if $a<b$. Conversely, if $\mu$ is a meaure on $\mathcal{B}$
  that is finite on bounded intervals then $F$ defined by $F(x) =
  \mu((-\infty, x), x > 0, F(0) = 0$ and $F(x) = - \mu(-x, 0]), x < 0$
  is increasing and normalized.
\end{thm}
\begin{proof}
  We begin by defining a function $\mu_*$ on the subsets of $\R$ via
  \[
  \mu_*(E) = \inf \sum_{j=1}^\infty (F(b_j) - F(a_j))
  \]
  where the infimum is taken over all coverings of $E$ of the form
  $\bigcup_{j=1}^\infty (a_j, b_j]$.

  It is straightforward to verify that $\mu_*$ is an outer measure on
  $\R$. We then observe that $\mu_*((a,b]) = F(b) - F(a)$ is
  $a<b$. Clearly, we see that $\mu_*((a,b]) \leq F(b) - F(a)$ because
  $(a, b]$ covers itself. Now suppose that $\bigcup_{j=1}^\infty
  (a_j,b_j]$ covers $(a,b]$. Then it must also cover $[a',b]$ for any
  $a < a' < b$. Because $F$ is continuous we see that for any
  $\epsilon >0$ we ccan always choose $b_j' > b_j$ such that $F(b_j')
  < F(b_j) + \epsilon/2^j$. The union of such intervals covers
  $[a',b]$. By the compactness of this interval we can see that only
  finitely many of these are needed. So $\bigcup_{j=1}^N(a_j,b_j')$
  covers $[a',b]$ for some $N$. Because $F$ is increasing we have that
  \[
  F(b) - F(a') \leq \sum_{j=1}^N F(b_j') - F(a_j) \leq \sum_{j=1}^N
  F(b_j) - F(a_j) + \epsilon/2^j) \leq \mu_*((a,b]) + \epsilon
  \]
  We then let $a' \to a$ and use the fact that $F$ is continuous to
  get that $F(b) - F(a) \leq \mu_*((a,b]) + \epsilon$. Because
  $\epsilon$ was arbitrary we have
  \[
  F(b) - F(a) = \mu_*((a,b])
  \]

  Now we show that $\mu_*$ is a metric exterior measure. Since $\mu_*$
  is an outer measure we have that $\mu_*(E_1,\cup E_2) \leq
  \mu_*(E_1) + \mu_*(E_2)$ so it suffices to see that the reverse
  inequality is true whenever $d(E_1,E_2) \geq \delta$ for some
  $\delta > 0$.

  Fix an $\epsilon >0$ and suppose that $\bigcup_j (a_j,b_j]$ is a
  covering of $E_1 \cup X_2$.
\end{proof}

\section{Differentiation Theory}

\section{Hilbert Spaces}

\section{Banach Spaces}

\section{Applications of Banach Space Theory}

\end{document}