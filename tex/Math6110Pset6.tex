\documentclass{article}
\usepackage{amsfonts,amsmath,amssymb,amsthm,relsize,fancyhdr,parskip,graphicx}

\pagestyle{fancy}
\lhead{Ben Carriel}
\chead{Math 6110 Problem Set 6}
\rhead{\today}

\parskip 7.2pt
\parindent 8pt

\DeclareMathOperator{\Z}{\mathbb{Z}}
\DeclareMathOperator{\Q}{\mathbb{Q}}
\DeclareMathOperator{\R}{\mathbb{R}}
\DeclareMathOperator{\capchi}{\raisebox{2pt}{$\mathlarger{\mathlarger{\chi}}$}}

\DeclareMathOperator{\divides}{\mathrel{|}}
\DeclareMathOperator{\suchthat}{\mathrel{|}}

\DeclareMathOperator{\lra}{\longrightarrow}
\DeclareMathOperator{\into}{\hookrightarrow}
\DeclareMathOperator{\onto}{\twoheadrightarrow}
\DeclareMathOperator{\bijection}{\leftrightarrow}

\newcommand{\problem}[1]{\noindent{\textbf{Problem #1}}\\}
\newcommand{\problempart}[1]{\noindent{\textbf{(#1)}}}

\newcommand{\der}[2]{\frac{\partial #1}{\partial #2}}

\newtheorem*{thm}{\\ Theorem}
\newtheorem*{lem}{\\ Lemma}
\newtheorem*{claim}{\\ Claim}
\newtheorem*{defn}{\\ Definition}
\newtheorem*{prop}{\\ Proposition}

\begin{document}
\problem{3.5.12}
It is clear by elementary calculus that the function
\[
F(x) = \begin{cases}
0                       & x = 0 \\ 
x^2\sin(1/x^2) & \text{otherwise}
\end{cases}
\]                                                                                                                                                                                                                                                                                                                                                                                                                                                                                                                                                                                                              
is differentiable everywhere with derivative
\[
F'(x) = \begin{cases}
0                                                          & x = 0 \\
2x\sin(1/x^2) - \frac{2\cos(1/x^2)}{x} & \text{otherwise}
\end{cases}
\]
We now need to show that $F'$ is not integrable on $[-1,1]$. First we note that $F'$ is symmetric about the origin so that 
\[
\int_{-1}^1 |F'(x)|dx = 2\int_{0}^1 |F'(x)|
\]
We will show that the integral on the right diverges. The intuition is that near the origin $\sin(1/x^2)$ and $\cos(1/x)$ oscillate too rapidly and add infinitely much ``area'' under the curve. More formally, consider the set of points 
\[
x_j = \left\{\frac{2}{(2j+1)\pi}\right\}_{j=1}^\infty
\]
It is clear that $x_j \to 0$ as $j \to \infty$ and so it suffices to show that
\[
\int_{x_N}^1 |F'(x)|dx \to \infty \text{ as } N\to\infty
\]
We will break the domain into smaller intervals of the form $[x_j,x_{j+1}]$ and observe that
\[
\int_{x_N}^1 = \sum_{j=1}^N \int_{x_{j+1}}^{x_{j}} |F'(x)|dx
\]
Then note that
\[
\int_{x_{j+1}}^{x_{j}} |F'(x)|dx = \int_{\sqrt{\frac{1}{2j\pi}}}^{x_j} F'(x) \pm \int_{x_{j+1}}^{\sqrt{\frac{1}{2j\pi}}} F'(x) = \pm 2\int_{\sqrt{\frac{1}{2j\pi}}}^{x_j} F'(x) 
\]
Where the sign is chosen to make the value positive. We then evaluate each term in the partial sum
\[
\int_{\sqrt{\frac{1}{2j\pi}}}^{x_j} F'(x) = F(x) \Big |_{\sqrt{\frac{1}{2j\pi}}}^{x_j} = \frac{1}{(2j+1)\pi} 
\]
So that 
\[
\int_{x_N}^1 |F'(x)| = \frac{4}{\pi} \sum_{j=1}^N \frac{1}{(2j+1)}
\]
Which diverges as $N \to \infty$ by comparison with the harmonic series. Hence, $F'$ is not integrable on $[-1,1]$.

\problem{3.5.16}
\problempart{a} 

\problempart{b}

\problem{3.5.17} We proceed similarly to the proof of Theorem 3.21. We note that 
\[
|(f * K_\epsilon)(x)| = \left| \int f(x-y)K_\epsilon(y)dy \right| \leq \int |f(x-y)||K_\epsilon(y)|dy
\]
We break the integral on the right up into 
\[
\int_{|y| < \epsilon} |f(x-y)||K_\epsilon(y)|dy + \sum_{k=1}^\infty \int_{|y| \in [2^{k-1}\epsilon, 2^k\epsilon]}|f(x-y)||K_\epsilon(y)|dy
\]

For the first term, we note that 
\[
\int_{|y| < \epsilon} |f(x-y)||K_\epsilon(y)|dy \leq A\epsilon^{-d}\int_{|x-y| < \epsilon}|f(y)|dy 
\]
We then note that $|x-y| < \epsilon$ describes precisely the ball with radius $\epsilon$ about the origin. As a result we see that 
\[
\sup_{\epsilon > 0} \int_{|x-y| < \epsilon} |f(y)| \leq m(B_\epsilon)\sup_{r \to 0} \frac{1}{m(B_r)} \int_{B_r} |f(y)|dy
\]
This gives 
\[
\int_{|y| < \epsilon} |f(x-y)||K_\epsilon(y)|dy \leq Ac_df^*(x)
\]
Where $c_d$ is a constant depending on the volume of the unit ball in $\R^d$. To estimate 
\[
\sum_{k=1}^\infty \int_{|y| \in [2^{k-1}\epsilon, 2^k\epsilon]}|f(x-y)||K_\epsilon(y)|dy
\]
We restrict our attention to the partial sums and observe that
\[
\int_{|y| \in [2^{k-1}\epsilon, 2^k\epsilon]}|f(x-y)||K_\epsilon(y)|dy \leq \frac{2^{(1-k)(d+1)}}{\epsilon^{d+1}}\int_{|y| \leq 2^k\epsilon} |f(x-y)|dy
\]
By the same reasoning as the first case (multiply by volume of appropriate ball and canceling powers), this gives
\[
2^d2^{1-k}\epsilon^{-1}c_df^*(x)
\]
As a result of this
\begin{align*}
\sum_{k=1}^\infty \int_{|y| \in [2^{k-1}\epsilon, 2^k\epsilon]}|f(x-y)||K_\epsilon(y)|dy &\leq \sum_{k=1}^\infty 2^d2^{1-k}\epsilon^{-1}c_df^*(x) \\
&\leq 2^{d+1}\epsilon^{-1}c_df^*(x)
\end{align*}
So in total we have that
\[
|(f * K_\epsilon)(x)| \leq (2^{d+2}c_dA)f^*(x)
\]
as desired. 

\problem{3.5.19}
\problempart{a} Let $E \subset \R$ have measure 0. Pick some $\epsilon > 0$ and use the absolute continuity of $f$ to see that 
\[
\sum_{k=1}^N |b_k - a_k| < \delta \text{ implies } \sum_{k=1}^N |f(b_k) - f(a_k)| < \epsilon
\] 
We then use the fact that $E$ has measure 0 to find some open set $\mathcal{O} \supset E$ such that $m(\mathcal{O}) < \delta$. Because $\mathcal{O} \subset \R$ we can write it as a disjoint union of open intervals $I_k = (a_k, b_k)$ such that
\[
\mathcal{O} = \bigcup_{k=1}^\infty I_k
\]
and then we note that 
\[
m(\mathcal{O}) = \sum_{k=1}^\infty (b_k - a_k) < \delta
\]
Moreover we have that the closure of $\mathcal{O}$ also satisfies $m(\overline{\mathcal{O}}) < \delta$ because the boundary of $\mathcal{O}$ is countable and hence has measure 0. Now we look at the images of these closed intervals under $f$. Consider the points $\{y_k\}_{1}^\infty$ and $\{Y_k\}_{1}^\infty$ which are defined to be the points such that $y_k = \inf_{x \in I_k} f(x)$ and $Y_k = \sup_{x \in I_k} f(x)$. Then set $\{x_k = \{f^{-1}(y_k)\}$ and $\{X_k = \{f^{-1}(Y_k)\}$. We have that $x_k,X_k \in \overline{I_k}$ because $f$ is continuous and $\overline{I_k}$ is compact. So then we have that
\[
f(\mathcal{O}) \subset \bigcup_k [f(x_k), f(X_k)]
\]
Because next note that
\[
\sum_{k=1}^\infty |X_k - x_k| \leq \sum_{k=1}^\infty |b_k - a_k| < \delta
\] 
because $x_k,y_k \in I_k$ and so the absolute continuity of $f$ implies that 
\[
m(\overline{\mathcal{O}}) = \sum_{k=1}^\infty |f(X_k) - f(x_k)| < \epsilon
\]
So 
\[
m(E) < m(\overline{\mathcal{O}}) < \epsilon
\] 
Letting $\epsilon \to 0$ gives the result. 

\problempart{b} Let $E$ now be a measurable subset of $\R$. Then we know that $E$ differs from an $F_\sigma$ by a set of measure 0. So we can write
\[
E = \left(\bigcup_{k=1}^\infty \mathcal{I}_k\right) \cup Z
\]
Where $Z$ is a set of measure 0 and each of the $I_k$ is closed (and bounded and hence, compact). Then the image under $f$ gives
\[
f(E) = \bigcup_k f(\mathcal{I}_k) \cup f(Z)
\]
The first element in the union is a countable union of measurable sets, and hence, measurable and $f(Z)$ has measure 0 by the previous part. Hence, $f(E)$ is measurable. 

\problem{3.5.20}
\problempart{a} The main hurdle to this exercise is that if $F' = 0$ on any interval, then $F$ is constant a.e. on that interval, and hence not strictly increasing. As as result, we are led to try to find a set with positive measure, but which contains no intervals, and then define $F$ as an integral over such a set. We constructed such a set (the ``fat'' Cantor set) in a previous problem set. Let $\mathcal{C}$ denote such a set (translated and dilated to appropriately sit in $[a,b]$). Consider the function
\[
D_\mathcal{C}(t) = d(t,\mathcal{C}) = \inf_{x\in \mathcal{C}} d(t,x)
\]
It is clear that $D_\mathcal{C}(t)$ is continuous and satisfies $D_\mathcal{C}(t) \geq 0$ with equality iff $t\in \mathcal{C}$. As a result, the function
\[
F(x) = \int_a^x D_\mathcal{C}(t)dt
\]
is well-defined. Moreover, $F$ is absolutely continuous on $[a,b]$ because of the absolute continuity of the integral. We can see that $F$ satisfies the requirements because $F' = 0$ a.e. in $\mathcal{C}$, which is a set of positive measure. To see that $F$ is strictly increasing, choose $x,y \in [a,b]$ with $x < y$, then we just set $y = x + h$ for some $h$ and note that 
\[
F(y) - F(x) = \int_a^{x+h} D_\mathcal{C}(t)dt - \int_a^{x} D_\mathcal{C}(t)dt = \int_x^{x+h} D_\mathcal{C}(t)dt
\] 
So we are integrating over some interval $I_x = [x,x+h]$. We then note that $\mathcal{C}$ contains no intervals and so there must be some interval $\mathcal{I}_x \subset \mathcal{C^c}$ such that
\[
\int_x^{x+h} D_\mathcal{C}(t)dt \geq \int_{\mathcal{I}_x} D_\mathcal{C}(t)dt > 0
\] 
Because $D_\mathcal{C}(t) > 0$ when $t \not\in \mathcal{C}$. Hence, $F$ is strictly increasing and we are done. 
 
\problempart{b} Define $F$ as in part $\bf{(a)}$. Then we have that $F$ is strictly increasing and absolutely continuous. We then set $\mathcal{O} = [a,b] - \mathcal{C}$. Then $\mathcal{O}$ is open, as it is the complement of a closed set and so we can write
\[
\mathcal{O} = \bigcup_{k=1}^\infty I_k
\]
Where the $I_k$ are a collection of disjoint open intervals. Because $F$ is increasing $I_j \cap I_k = \emptyset$ implies that $F(I_j) \cap F(I_k) = \emptyset$ and so we have that
\[
F(\mathcal{O}) = \bigcup_{k=1}^\infty F(I_k)
\] 
And so we have that
\[
m(\mathcal{O}) = \sum_{k=1}^\infty m(I_k) = \int_{\mathcal{O}}D_\mathcal{C}(t)dt
\]
We then note that $\int_\mathcal{C} D_\mathcal{C}(t)dt = 0$ so that
\[
\int_\mathcal{O} d_\mathcal{C}(t)dt = \int_{[a,b]} D_\mathcal{C}(t)dt = F(b) - F(a)
\]
So we have that $m(F(\mathcal{O})) = m(F([a,b]))$ and as a result $m(F(\mathcal{C})) = 0$. Consequently, if we take any other set $U \subset \mathcal{C}$ we must have that $m(F(U)) = 0$. However, we also have that because $m(\mathcal{C}) > 0$ that it contains a non-measurable subset, $\mathcal{N}$. But then $F(\mathcal{N}) \subset F(\mathcal{C})$ and so $m(F(\mathcal{N})) = 0$ however $F^{-1}(F(\mathcal{N})) = \mathcal{N}$ is non-measurable and we are done. 

% TODO % 
\problempart{c}

\problem{3.5.21}
\problempart{a}

\problempart{b} Before we begin we need to prove the following fact
\begin{lem}
Let $f$ and $g$ absolutely continuous. Then
\[
(f\circ g)'(x) = (f' \circ g)(x)g'(x)
\]
almost everywhere. 
\end{lem}
%TODO Add in the reason that abs cont means measure 0 maps to measure 0
\begin{proof}
Because both functions are absolutely continuous we must have that they are differentiable almost everywhere. As a result, we can consider the set $D$ where both $f'$ and $g'$ exist (it's complement has measure 0). Let
\[
\Delta_h(u) = u(x+h) - u(x)
\]
Then we note that
\[
\Delta(f \circ g) = [(f \circ g) + o(\Delta(g))]\cdot \Delta_t(g)
\]
Then divide by We then divide by $t$ giving and apply the definition of the derivative to see that
\[
(f \circ g)'(x) = (f' \circ g)(x)g'(x)
\]
on $D$. And as a result, almost everywhere so we are done. 
\end{proof}
To complete the proof we define the following function
\[
h(x) = \int_{-\infty}^x f(t)dt 
\]
We know $h$ is absolutely continuous because $f$ is integrable. We then compute
\[
\int_a^b (h\circ F)'(x)dx = (h \circ F)(b) - (h\circ F)(a) = \int_{-\infty}^{F(b)} f(y)dy -\int_{-\infty}^{F(a)} f(y)dy = \int_{A}^{B} f(y)dy 
\]
Applying the lemma to this identity gives the desired result. 
\end{document}